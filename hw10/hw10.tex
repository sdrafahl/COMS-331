    \documentclass[11pt]{article}
    %	options include 12pt or 11pt or 10pt
    %	classes include article, report, book, letter, thesis
    
    \title{HW10}
    \author{Shane Drafahl}
    \date{16 October ,2017}
    \usepackage{graphicx}
    \usepackage{epstopdf}
    \usepackage{graphics}

    \begin{document}
    \maketitle

     1. For union for $ L_{1} $ and $ L_{2} $ I have machines $ M_{1} $ and $ M_{2} $. Given 
     an input $ x \in L_{1} \cup L_{2} $ I give x to $ M_{1} $ and $ M_{2} $ and if its accepted by
     either machine that means they are accepted. Similar to intersection but both machines must be accepted.
     I would build this turring machine by getting two tape and with input x and then dove tailing the 
     two tape into each relative turring machine. For example below is a representation of the tape
     where $ b_{1} = a_{1} $ and so on and so forth.

     $ \newline $

     So for union 

     $ \newline $

     \begin{verbatim}

     input x in Sigma*
     Run M1 on x and then run M2 on x 
     Accept if M1 or M2 both accept

     \end{verbatim}

      $ \newline $

      intersection

      $ \newline $

     \begin{verbatim}

     input x in Sigma*
     Run M1 on x and then run M2 on x 
     Accept if M1 and M2 both accept

     \end{verbatim}

      Reversal

      $ \newline $

     \begin{verbatim}

     input x in Sigma*
     Assign y:= reverse(x)
     Run M1 on x and then run M2 on y 
     Accept if M1 and M2 both accept

     \end{verbatim}


    $ \newline $

    For reversal I would simply copy the input from one side of the tape to the other in reverse
    order. I would then have the same turring machine dove tail both peices of tape and if both are 
    accepted then the whole thing is accepted.

    $ \newline $

    Reversal

      $ \newline $

     \begin{verbatim}

     input x in Sigma*
     Assign y:= reverse(x)
     Run M1 on x and then run M2 on y 
     Accept if M1 and M2 both accept

     \end{verbatim}

    2. Given a string that z = xy and you have $ M_{1} $, $ M_{2} $. You create every possible combination
    of ways to divided up z onto multiple peices of tape. For example if x = ab and y = cd then
    you have 
    $ \newline $
    $ T[a]_{1} ...  T[bcd]^{2} $ 
    $ \newline $
    $ T[ab]_{3} ... T[cd]_{4} $
    $ \newline $
    $  T[abc]_{5} ... T[d]_{6} $
    $ \newline $
    Where $ T[]_{n} $ is a peice of tape. $ M_{1} $ dove tails for every odd indexed tape and $ M_{2} $
    does every even. If both of the machines have at least one accepting string then the whole thing is
    accepted. 


    $ \newline $

    concatination

      $ \newline $

     \begin{verbatim}

     input x in Sigma*
     Assign t1[], t2[]:= sub(x) // returns every possible sub string of x to arrays of tape for t1 and t2
     Run M1 for all t1 and run M2 for all t2
     Accept if M1 and M2 have at least Accept one string each.

     \end{verbatim}

     $ \newline $

    3. a 

    $ \newline $

    For this turring machine it would depend on if it goes left,right, or does nothing
    $ \newline $
    $ \delta (a_{n}, a_{n}) $ = $ (a_{n + 1}, R) $ and push  (GO RIGHT)
    $ \newline $
    $ \delta (a_{n}, a_{n}) $ = $ (a_{n - 1}, L) $ (GO LEFT)
    $ \newline $
    $ \delta (a_{n}, a_{n}) $ = $ (a_{n}, a_{n}) $ (Do Nothing)
    $ \newline $
    b. 
    $ \newline $
    
    To prove that all languages accepted by PA's can be accepted by M we will prove that the accepted
    languages of $ PA \subset M $. We can prove this because we can simulate a PA with M by starting
    the head on the left most data on the read only tape and only using $ \Gamma_{1} $ and reading
    from left to right.  

    $ \newline $

    We can prove that it is equivalant to a turring machine because $ M \subset T $ and $ T \subset M $.
    In this case we can prove $ M \subset T $ because a Turring machine can use two peices of tape to
    simulate the two stacks and a third as the read only tape. We can also prove $ T \subset M $.
    We can simulate a turring machine because we can put all the contents left of the head in $ \Gamma_{1} $.
    and to the right $ \Gamma_{2} $. If we want to move right we pop $ \Gamma_{2} $ and left $ \Gamma_{1} $.
    Therfore they are equal. We also know that all languages accepted by any type of PA including
    NPDA or determinsitic ones can also be accepted by a turring machine so that is another reason 
    M can accept the same languages as a class of automata.

    



    \end{document}