\documentclass[11pt]{article}
%	options include 12pt or 11pt or 10pt
%	classes include article, report, book, letter, thesis

\title{HW13}
\author{Shane Drafahl}
\date{16 October ,2017}
\usepackage{graphicx}
\usepackage{epstopdf}
\usepackage{graphics}

\begin{document}
\maketitle

1. $ G = (\{\ G,T,B,A \}\, \{\ 1,0 \}\ ,S,P) $
$ \newline $
$ P = \{\ $
$ \newline $
$ S \rightarrow GTB | \epsilon $
$ \newline $
$ G \rightarrow GA | GC |\epsilon $
$ \newline $
$ A1 \rightarrow 1A $
$ \newline $
$ A0 \rightarrow 0A $
$ \newline $
$ AT \rightarrow 1TA $
$ \newline $
$ C0 \rightarrow 0C $
$ \newline $
$ C1 \rightarrow 1C $
$ \newline $
$ AB \rightarrow 1B $
$ \newline $
$ CB \rightarrow 0A $
$ \newline $
$ CT \rightarrow 0TC $
$ \newline $
$ \}\ $

$ \newline $
2. $ L = \{\ p(M)p(w) : $ M uses a finite number of tape cells when running on input w $ \}\ $
$ \newline $

To show that L is not decidable. We will use proof by contradiction and assume L is
a decidable language. Therefore there would exist a Turring Machine $ M_{L} $ that can decide L.
We can use $ M_{L} $ as a subroutine to construct a decider for the halting problem.
Suppose we construct $ M_{k} $ that takes $ k $.

$ \newline $

$ M_{k} $ : Input $ x \in \Sigma^{*} $
$ \newline $
1. If x is not of the form

$ \newline $

Otherwise another way to prove this is the complement of L is a language where
M uses an infinite number of tape cells. This is not acceptable because this machine
can never say yes because it will be infinitely checking tape cells for a yes. Therefore
becaues the complement is not turring acceptable it cannot be decidable.

$ \newline $

$ L = \{\ p(M)p(w)01^{n}0 : $ M uses at most n tape cells when running on input w $ \}\ $

$ \newline $

This language is turring decidable. This language is acceptable because if it uses at most n tape
cells it can give a Y. To prove its turring decidable we will look at all the cases.
$ \newline $
Case 1: $ M \searrow w $ in this case you just have to count the number of states that the head goes through. If it
uses more than n then M says no and if it uses less or n then M says yes.

$ \newline $

Case 2: $ M \searrow w $ there are two ways this can happen. It can either run forever over a finite amount of states
or the head can transition over an infinite set of states. If it transitions over a finite number of states then
M could count the number of times the head transitions into each state and put a limit on the number of times it can enter the same state
multiple times. This prevents looping over a finite number of states forever. After it detects looping if it has gone through
more than n transitions M can say no otherwise it can say yes. If M loops over an infiite number of transition then at some point
it would reach N an pass it so as soon as it goes over N it can say no. So therefore this machine can always produce a yes or no so it is decidable.




\end{document}
