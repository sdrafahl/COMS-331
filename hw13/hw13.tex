\documentclass[11pt]{article}
%	options include 12pt or 11pt or 10pt
%	classes include article, report, book, letter, thesis

\title{HW13}
\author{Shane Drafahl}
\date{16 October ,2017}
\usepackage{graphicx}
\usepackage{epstopdf}
\usepackage{graphics}

\begin{document}
\maketitle

1. $ G = (\{\ S,A,B \}\, \{\ 1,0 \}\ ,S,P) $
$ \newline $
$ P = \{\ $
$ \newline $
$ S \rightarrow ASA | BSB | \epsilon $
$ \newline $
$ A \rightarrow 1 $
$ \newline $
$ B \rightarrow 0 $
$ \newline $
$ \}\ $

$ \newline $
2. $ L = \{\ p(M)p(w) : $ M uses a finite number of tape cells when running on input w $ \}\ $
$ \newline $
This is turring acceptable not but turing decidable. It is turring acceptable because
if it halts it will be a yes. It is not decidable and I will use reduction to prove this.
I will reduce the halting problem to M that decides L.
Suppose for contradiction that L is a decidable language for all w. If M
for L says yes then we know it halts. Othwerwise it does not halt because if it processed
an infinite number of tape cells then it would never stop and reach the halting state.
Therefore the halting problem can be reduced to M for L and so therefore
L is not a decidable language.

$ \newline $

Otherwise another way to prove this is the complement of L is a language where
M uses an infinite number of tape cells. This is not acceptable because this machine
can never say yes because it will be infinitely checking tape cells for a yes. Therefore
becaues the complement is not turring acceptable it cannot be decidable.

$ \newline $

$ L = \{\ p(M)p(w)01^{n}0 : $ M uses at most n tape cells when running on input w $ \}\ $

$ \newline $

This language is turring decidable. This language is acceptable because if it uses at most n tape
cells it can give a Y. To prove its turring decidable we will look at all the cases.
$ \newline $
Case 1: $ M \searrow w $ in this case you just have to count the number of states that the head goes through. If it
uses more than n then M says no and if it uses less or n then M says yes.

$ \newline $

Case 2: $ M \nearrow w $ there are two ways this can happen. It can either run forever over a finite amount of states
or the head can transition over an infinite set of states. If it transitions over a finite number of states then
M could count the number of times the head transitions into each state and put a limit on the number of times it can enter the same state
multiple times. This prevents looping over a finite number of states forever. After it detects looping if it has gone through
more than n transitions M can say no otherwise it can say yes. If M loops over an infiite number of transition then at some point
it would reach N an pass it so as soon as it goes over N it can say no. So therefore this machine can always produce a yes or no so it is decidable.




\end{document}
