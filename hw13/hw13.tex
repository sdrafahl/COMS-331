\documentclass[11pt]{article}
%	options include 12pt or 11pt or 10pt
%	classes include article, report, book, letter, thesis

\title{HW13}
\author{Shane Drafahl}
\date{16 October ,2017}
\usepackage{graphicx}
\usepackage{epstopdf}
\usepackage{graphics}

\begin{document}
\maketitle

1. $ G = (\{\ S,A,B \}\, \{\ 1,0 \}\ ,S,P) $
$ \newline $
$ P = \{\ $
$ \newline $
$ S \rightarrow ASA | BSB | \epsilon $
$ \newline $
$ A \rightarrow 1 $
$ \newline $
$ B \rightarrow 0 $
$ \newline $
$ \}\ $

$ \newline $
2. $ L = \{\ p(M)p(w) : $ M uses a finite number of tape cells when running on input w $ \}\ $
$ \newline $
This language is turring acceptable. For $ \forall_{w} \in L $ every $ w $ will only take finite number of tape cells
so we know it will halt when the head of M reaches the last tape cell so $ \forall_{w} \in L $, $ M \searrow w $.
So therefore L is acceptable. L is not a decidable language. We can prove this by reduction. We
want to Reduce L to the halting problem. Suppose by contradiction we assume L is decidable.
This means that there exists a machine M that will say yes to strings in the language. Othwerwise if the string
is not in the language it will say no. If a string requires an infinte number of cells
while running on the string then it cannot halt because it cannot process more cells after it halts. So
if a string x is not in L then it will not halt. Therefore if a string is in L it will halt otherwise it
will not halt if it is not in L. Our assumption was that there exists a machine that decides L but this is a
contradiction because that would mean that a machine can decide the halting problem. Therefore L is
not decidable.

$ \newline $

Otherwise another way to prove this is the complement of L is a language where
M uses an infinite number of tape cells. This is not acceptable because this machine
can never say yes because it will be infinitely checking tape cells for a yes. Therefore
becaues the complement is not turring acceptable it cannot be decidable.

$ \newline $

$ L = \{\ p(M)p(w)01^{n}0 : $ M uses at most n tape cells when running on input w $ \}\ $

$ \newline $

This language is turring decidable. This language is acceptable because if it uses at most n tape
cells it can give a Y. To prove its turring decidable we will look at all the cases.
$ \newline $
Case 1: $ M \searrow w $ in this case you just have to count the number of states that the head goes through. If it
uses more than n then M says no and if it uses less or n then M says yes.

$ \newline $

Case 2: $ M \nearrow w $ there are two ways this can happen. It can either run forever over a finite amount of states
or the head can transition over an infinite set of states. If it transitions over a finite number of states then
M could count the number of times the head transitions into each state and put a limit on the number of times it can enter the same state
multiple times. This prevents looping over a finite number of states forever. After it detects looping if it has gone through
more than n transitions M can say no otherwise it can say yes. If M loops over an infiite number of transition then at some point
it would reach N an pass it so as soon as it goes over N it can say no. So therefore this machine can always produce a yes or no so it is decidable.




\end{document}
