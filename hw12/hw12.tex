\documentclass[11pt]{article}
%	options include 12pt or 11pt or 10pt
%	classes include article, report, book, letter, thesis

\title{HW11}
\author{Shane Drafahl}
\date{16 October ,2017}
\usepackage{graphicx}
\usepackage{epstopdf}
\usepackage{graphics}

\begin{document}
\maketitle

1. You are employed as a programmer, and you are asked to write a program that:

$ \newline $

(a) receives in input a generic C program x, and counts the number of statements in x.

$ \newline $

Y

$ \newline $

(b) receives in input a generic C program x and an input string w, and counts the number
of statements executed at least once when x runs on input w.

$ \newline $

N

$ \newline $

(c) receives in input a generic C program x and an input string w, and counts the number
of statements never executed when x runs on input w

$ \newline $

N

$ \newline $

(d) receives in input a generic C program x and decides whether x is syntactically correct.

$ \newline $

Y

$ \newline $

(e) receives in input two natural numbers and computes a specific function $ f: N^{2} \rightarrow N $

$ \newline $

N

$ \newline $

(f) receives in input a generic arithmetic expression e composed of integers and the four
arithmetic operators, and computes its value.

$ \newline $

Y

$ \newline $

(g) halts on the empty string.

$ \newline $

Y

$ \newline $

(h) receives in input a generic C program x and decides whether x halts only on the empty
string.

$ \newline $

N

$ \newline $

(i) receives in input two generic regular expressions and decides whether they are equivalent.

$ \newline $

Y

$ \newline $

(j) receives in input a generic C program x and the name of one of its functions, f, and
decides whether x can ever call f.

$ \newline $

N

$ \newline $

(k) receives in input a generic C program x, an input string w, and the name of one of its
functions, f, and decides whether x calls f when runnning on input w.

$ \newline $

N

$ \newline $

(l) receives in input two generic C programs $ x_{1} $ and $ x_{2} $ and an input string w, and decides
whether $ x_{1} $ and $ x_{2} $ produce the same output when running on input w.

$ \newline $

N

$ \newline $

(m) receives in input two generic C programs x1 and x2, and decides whether x1 and x2
produce the same output when running on every possible input.

$ \newline $

N

$ \newline $

(n) receives in input two generic C programs x1 and x2, and decides whether x1 and x2
produce the same output when running on at least one input.

$ \newline $

N

$ \newline $

(o) receives in input a generic C program x, an input string w, and a natural number n, and
decides whether x uses less than n bytes of memory when running on input w.

$ \newline $

Y

$ \newline $

(p) receives in input a generic C program x, an input string w, and decides whether there is
$ n \in N $ such that x uses less than n bytes of memory running on input w.

$ \newline $

N

$ \newline $

2. Use reduction to prove that the language

$ \newline $

Suppose by contradiction that L is decidable. We will reduce the halting problem to L. To reduce we will create a new machine
that is the same as the halting problem with a machine that takes L. We will assume that machine $ M^{'} $ takes L. If the machine $ M $
with input $ w $ enters its inital state more than once then if it was given to $ M^{'} $ it would return true. Suppose that
the machine we are making is $ M^{''} $. We will create $ M^{''} $ with $ M^{'} $. In order to do this $ M^{''} $ will append a transition
to the initial state to $ w $. In otherwords for all inputs to $ M^{'} $ if it halts it will be true. If $ M^{'} $ does not halt
it will never return to the initial state and since we assume its decidable it would be false. This is a contradiction because
this machine $ M^{''} $ is the halting problem because it returns true if it halts and false if it does not. Therefore by reduction L
is not decidable.



\end{document}
