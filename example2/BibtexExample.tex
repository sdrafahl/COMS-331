\documentclass{article}

\title { Symbolic state space generation for \\
         finite models with unknown bounds
}
\author{ Gianfranco Ciardo ~~~~ Robert Marmorstein ~~~~ Radu Siminiceanu\\[2ex]
         College of William and Mary,
         Williamsburg, Virginia 23187  \\
         {\rm \{ciardo,rmmarm,radu\}@cs.wm.edu} }

\date{}
\begin{document}
\maketitle

\begin{abstract}
In previous work, we proposed a ``saturation'' algorithm
for symbolic state-space generation based on
multi-valued decision diagrams and boolean Kronecker operators.
This approach greatly outperforms traditional BDD-based techniques but, 
like them, assumes a priori knowledge of the state space of each submodel.
In this work, we introduce a new algorithm that merges explicit
local state-space discovery with symbolic global state-space generation.
This relieves the modeler from worrying about the behavior of submodels
in isolation.
\end{abstract}

\section{Introduction}\label{SEC:introduction}

Since the introduction of implicit methods in verification and symbolic
model checking, decision diagram, in particular
BDDs \cite{Bryant1986,Bryant1992,Burch1992}, have had enormous success. 
However, the systems targeted have been mainly VLSI and
protocols, where the possible values of the state variable is easily
determined a priori.
When attempting to use these same techniques to verify arbitrary
systems modeled in a high-level specification language such as
Petri nets or pseudocode, determining the range of the state variables is 
much more difficult.
Traditionally, the user has had to deal with this difficulty.
In NuSMV \cite{Cimatti1999NuSMV}, the domain for multi-valued variables
must be given explicitly as a set or integer range.
In our own work, the input (a Petri net) must be
partitioned so that the state space of
each resulting subnet can be generated in isolation;
in practice, the modeler is forced to carefully add
inhibitor arcs or other constraints to the net to realize this property.
Putting this burden in the user is limiting and possibly error-prone.

We tackle this problem with an algorithm that, with minimal overhead,
merges explicit exploration of the local state space of each submodel with
symbolic exploration of the global state space to produce a 
multi-valued decision diagram (MDD) \cite{Kam1998} representation of the 
final state-space, together with the exact representation of the local 
sub-spaces.  The algorithm is based on our efficient saturation algorithm
\cite{GFC-Saturation-2001}, which uses an MDD to store the states and 
boolean Kronecker matrices to encode the transition relation.
By breaking the traditionally monolithic transition function into 
event-based functions, exploiting \emph{event locality} 
\cite{GFC-MDD-improv-2000}, and performing \emph{in-place updates} of MDD 
nodes, the saturation algorithm showed massive time and space improvements 
over previously-known approaches. Our new algorithm also employs these 
ideas and is more general, as it can be applied to models where the 
traditional algorithm would fail.

\bibliographystyle{abbrv}
\bibliography{BibtexExample}

\end{document}
