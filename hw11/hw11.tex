    \documentclass[11pt]{article}
    %	options include 12pt or 11pt or 10pt
    %	classes include article, report, book, letter, thesis
    
    \title{HW11}
    \author{Shane Drafahl}
    \date{16 October ,2017}
    \usepackage{graphicx}
    \usepackage{epstopdf}
    \usepackage{graphics}

    \begin{document}
    \maketitle

    1. 
    $ \newline $
    $ \footnotesize L = \{\ p(M)p(w):$ M uses a finite number of tape cells when running on input w $ \}\ $
    $ \newline $
    This language is acceptable because it only has a finite number of tape so even if the input is greater than
    then memmory storage it can only hold a finite number from the input. This language is not decidable because
    if M needs to process info from the whole input w, if there are w/2 finite memmory on the tape it can
    never process the whole tape. 

    $ \newline $

    $ \footnotesize L = \{\ p(M)p(w)01^{n}0:$ M uses at most n tape cells when running on input w $ \}\ $

    $ \newline $
    
    This language is acceptable but not decidable. The turring machine input may not be able to store the
    entire input. $ p(M)p(w)01^{n}0 $ = $ 01^{m}001^{}0 $ where m is the number of instructions for the head
    for machine $ M $. We can use $ n $ at most tape cells so we cant get stuck in a infinite for loop
    because its finite but we cannot read both machines that are deliminated by 0's so it can not be decidable.

    $ \newline $
    
    2.

    $ \newline $

    $ \{\ p(M) : |L(M)| \leq 10 \}\ $

    $ \newline $

    This language is turing-decidable because the languge has a cardinality less than or equal 
    to ten. The turing machine could simply have a finite number of states to represent every possible
    string in $ L(M) $ and either print a Y or a N based on those strings.

    $ \newline $

    $ \{\ p(M) : |L(M)| \geq 10 \}\ $

    $ \newline $

    This language is not acceptable because the language is some length greater than 10. 
    If the turring machine does not halt we do not know if its looping forever or if it has just not
    reached the end of the language yet. 

    $ \newline $

    $ \{\ p(M)p(w) : M \searrow w $ in 10 steps or less $ \}\ $

    $ \newline $

    This is turing decidable because it has a finite number of steps so we know for sure it 
    halts. If in 10 steps it has not printed a Y then it would print a N by the last step.

    $ \newline $

    $ R = \{\ p(M)p(w) : M \searrow w $ in 10 steps or more $ \}\ $

    $ \newline $

    This language halts so we know it is acceptable. We will use reduction to prove it is not
    decidable. We know from the class notes that $ \{\ p(M)p(w) : M \searrow w \}\ $. 
    If we build a machine $ M^{'} $. This machine first takes input $ w $ and checks
    that it is at least 10 steps. If not it adds more steps to the tape. This tape is then given
    to $ R $. If $ R $ returns a Y or a N then $ M^{'} $ returns a Y or a N otherwise it loops.
    So therefore $ R $ reduces to $ \{\ p(M)p(w) : M \searrow w \}\ $ so the language is only 
    accepting by not decidable.

    $ \newline $

    3. Use reduction to prove that the language is not decidable.

    $ \newline $

    $ L = \{\ p(M_{1})p(M_{2}) : L(M_{1}) \subseteq L(M_{2}) \}\ $
    
    $ \newline $

    $ L(M_{1}) \subseteq L(M_{2}) $ so therefore $ M_{2} \searrow p(M_{1}) $ because 
    every word in $ p(M_{1}) $ is also in $ L(M_{2}) $. We know that 
    $ R = \{\ p(M)p(w) : M \searrow w \}\ $ is not decidable so we need to reduce L into it.
    $ M_{2} \searrow p(M_{1}) $ So essentialy we just need to take the encoding for 
    $ M_{2} $ and simulate that turing machine and take $ p(M_{2}) $ as input. We can say
    this input is $ p(M_{2}) = w $. This is the same as $ \{\ p(M)p(w) : M \searrow w \}\ $
    where $ p(w) $ is the encoding of $ P(M_{1}) $ and $ P(M) $ is $ p(M_{2}) $ or the turing
    machine that takes the input. Therefore L reduces down to $ \{\ p(M)p(w) : M \searrow w \}\ $.

    \end{document}