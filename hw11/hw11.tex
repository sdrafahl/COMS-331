    \documentclass[11pt]{article}
    %	options include 12pt or 11pt or 10pt
    %	classes include article, report, book, letter, thesis
    
    \title{HW11}
    \author{Shane Drafahl}
    \date{16 October ,2017}
    \usepackage{graphicx}
    \usepackage{epstopdf}
    \usepackage{graphics}

    \begin{document}
    \maketitle

    1. 
    $ \newline $
    $ \footnotesize L = \{\ p(M)p(w):$ M uses a finite number of tape cells when running on input w $ \}\ $
    $ \newline $
    This language is acceptable but not decidable. It is not decidable because the size of the tape is unknown.
    We can create a turing machine that simply gives M the input w and L and check that it used a finite number of tape
    cells. Essentialy $ p(M)p(w) $ is the same as the problem $ \{\ P(M)p(w) : M \searrow w \}\ $ which we know is not decidable.

    $ \newline $

    $ \footnotesize L = \{\ p(M)p(w)01^{n}0:$ M uses at most n tape cells when running on input w $ \}\ $

    $ \newline $
    
    This language is decidable because we know the number n of tape cells that you can check. If it loops we can check 
    if the head goes beyond n cells. If it loops over the same cells we can count the number of times it goes over the cells.

    $ \newline $
    
    2.

    $ \newline $

    $ \{\ p(M) : |L(M)| \leq 10 \}\ $

    $ \newline $

    This language is not turing acceptable. The turing machine might go forever to comfirm that there are 10 strings
    or less in the machine. 

    $ \newline $

    $ \{\ p(M) : |L(M)| \geq 10 \}\ $

    $ \newline $

    This language is acceptable because if there is at least 10 words in the language it can be accepted. 
    This is not decidable because its complement is not turing acceptable.

    $ \newline $

    $ \{\ p(M)p(w) : M \searrow w $ in 10 steps or less $ \}\ $

    $ \newline $

    This is turing decidable because it has a finite number of steps so we know for sure it 
    halts. If in 10 steps it has not printed a Y then it would print a N by the last step.

    $ \newline $

    $ R = \{\ p(M)p(w) : M \searrow w $ in 10 steps or more $ \}\ $

    $ \newline $

    This language halts so we know it is acceptable. We will use reduction to prove it is not
    decidable. We know from the class notes that $ \{\ p(M)p(w) : M \searrow w \}\ $. 
    If we build a machine $ M^{'} $. This machine first takes input $ w $ and checks
    that it is at least 10 steps. If not it adds more steps to the tape. This tape is then given
    to $ R $. If $ R $ returns a Y or a N then $ M^{'} $ returns a Y or a N otherwise it loops.
    So therefore $ R $ reduces to $ \{\ p(M)p(w) : M \searrow w \}\ $ so the language is only 
    accepting by not decidable.

    $ \newline $

    3. Use reduction to prove that the language is not decidable.

    $ \newline $

    $ L = \{\ p(M_{1})p(M_{2}) : L(M_{1}) \subseteq L(M_{2}) \}\ $
    
    $ \newline $

    We know that it is not turing decidable for a turing machine to determine if two sets are equal. Using reduction 
    we will take a turing machine $ L $ that checks if two sets are equal. This machine $ L $ needs to check that 
    $ L(M_{1}) \subseteq L(M_{2}) $ and $ L(M_{2}) \subseteq L(M_{1}) $. We know that $ L $ can be reduced 
    to $ L(M_{1}) \subseteq L(M_{2}) $ and we know this cannot be decidable. If it could be decidable then 
    it would mean $ L $ would be decidable and that would be a contradiction. Therefore 
    $ L = \{\ p(M_{1})p(M_{2}) : L(M_{1}) \subseteq L(M_{2}) \}\ $ is not decidable.

    \end{document}